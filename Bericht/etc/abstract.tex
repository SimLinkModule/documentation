%!TEX root = ../Studienarbeit.tex

\pagestyle{empty}

\newenvironment{abstractpage}
  {\cleardoublepage\vspace*{\fill}\thispagestyle{empty}}
  {\vfill\cleardoublepage}
\newenvironment{abstractsection}[1]
  {\bigskip
   \begin{center}\bfseries#1\end{center}}
  {\par\bigskip}

\begin{abstractpage}
    \begin{abstractsection}{Kurzfassung}
      Ein wichtiger Bestandteil neben dem Multikopterfliegen stellt für Renn- und Freestyle-Quadro\-kopterpiloten das Training dar, welches in zwei Varianten durchgeführt werden kann. Einerseits am Flugplatz -- wobei durch Abstürze hohe Reparaturkosten entstehen können. Andererseits kostengünstig im Simulator -- am Rechner. Während des Trainings am Simulator ist es wünschenswert die gewohnte Fernsteuerung zu verwenden, um den Piloten eine gewohnte Umgebung zu bieten. Um dies zu erreichen, wird in dieser Arbeit ein \acs{BLE}-Erweiterungsmodul für Multikopterfernsteuerungen entwickelt, womit der Pilot ohne weitere Treiber die Fernsteuerung mit einem Simulator auf einem Endgerät verbinden kann. Damit die Verbindung ohne Treiber funktioniert, werden die Fernsteuerungsdaten mittels \acs{HOGP} verpackt. Ebenso werden Platinen konstruiert, mit denen das Erweiterungsmodul kompakt, mobil und benutzerfreundlich verwendet werden kann. Damit die entwickelten Platinen fest und kompakt im Modulschacht (Typ: Lite) befestigt werden können, wird zusätzlich noch ein Gehäuse entwickelt, welches mittels eines \acs{FDM}-3D-Druckers gedruckt werden kann. Zuletzt wird noch die Latenz zwischen der integrierten \acs{USB}-Verbindung der Fernsteuerung und der \acs{BLE}-Verbindung des Erweiterungsmoduls verglichen. Dabei kann festgestellt werden, dass unter optimalen Bedingungen die \acs{BLE}-Latenz durchschnittlich 20,84~ms größer ist als die \acs{USB}-Latenz. 
    \end{abstractsection}

    \begin{abstractsection}{Abstract}
      In addition to multicopter flying, an important aspect for racing and freestyle quadrocopter pilots is training, which can be carried out in two ways. On the one hand, at the airfield -- where crashes can result in high repair costs. On the other hand, inexpensive in the simulator -- on a computer. During training on the simulator, it is desirable to use the familiar radio control in order to provide the pilots with a familiar environment. To achieve this, a \acs{BLE} extension module for multicopter radio controls is developed in this project, which allows the pilot to connect the radio control to a simulator on a remote device without additional drivers. To ensure that the connection works without drivers, the radio control data is packaged using \acs{HOGP}. In addition, circuit boards are being designed so that the extension module can be used in a compact, mobile and user-friendly fashion. So that the developed circuit boards can be fixed firmly and tightly in the module slot (type: Lite), a shell is developed, which can be printed using an \acs{FDM} 3D printer. Finally, the latency between the integrated \acs{USB} connection of the remote control and the \acs{BLE} connection of the expansion module is compared. It can be measured that under optimal conditions, the \acs{BLE} latency is on average 20.84~ms longer than the \acs{USB} latency. 
    \end{abstractsection}
\end{abstractpage}