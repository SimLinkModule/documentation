%!TEX root = ../Studienarbeit.tex

\pagestyle{empty}

\newenvironment{abstractpage}
  {\cleardoublepage\vspace*{\fill}\thispagestyle{empty}}
  {\vfill\cleardoublepage}
\newenvironment{abstractsection}[1]
  {\bigskip
   \begin{center}\bfseries#1\end{center}}
  {\par\bigskip}

\begin{abstractpage}
    \begin{abstractsection}{Kurzfassung}
      Ein wichtiger Bestandteil, neben dem Multikopterfliegen, stellt für Renn- und Freestyle-Quadrokopterpiloten das Training dar. Dabei kann das Training in zwei Varianten durchgeführt werden. Einerseits das Training am Flugplatz, wobei hierbei durch Abstürze hohe Repaturkosten enstehen können, oder durch das Training im Simulator am Rechner. Während des Trainings am Simulator ist es wünschenswert die gewohnte Fernsteuerung zu verwenden, um den Piloten eine gewohnte Umgebung zu bieten. Um dies zu erreichen wird in dieser Arbeit ein \acs{BLE}-Erweiterungsmodul für Multikopterfernsteuerungen entwickeln, womit der Pilot ohne weitere Treiber die Fernsteuerung mit einen Endgerät verbinden kann. Damit die Verbindung ohne Treiber funktioniert, werden die Fernsteuerungsdaten mittels \acs{HOGP} verpackt sind. Ebenso werden Paltinen erstellt, womit das Erweiterungsmodul kompakt, mobil und benutzerfreundlich verwendet werden kann. Damit die entwickelten Paltinen kompakt und fest im Modulschacht (Typ: Lite) befestigt werden können, wird zusätzlich noch ein Gehäuse entwickelt, welches mittels eines \acs{FDM}-3D-Druckers gedruckt werden kann. Zuletzt wird noch die Latenz zwischen der intergrierten USB-Verbindung der Fernsteuerung und der \acs{BLE}-Verbindung des Erweiterungsmoduls verglichen. Dabei kann festgestellt werden, dass unter optimalen Bedingungen die \acs{BLE}-Latenz durchschnittlich 20,84~ms höher ist als die USB-Latenz. Für den produktiven Einsatz des Erweiterungsmoduls unter iOS- und iPadOS-Geräten muss jedoch noch die Zertifizierung durch das \acs{MFi}-Programm von Apple stattfinden. 
      \todo[inline]{TODO: Zeiten anschauen}
    \end{abstractsection}

    \begin{abstractsection}{Abstract}
      \todo[inline]{TODO: Abstract}
    \end{abstractsection}
\end{abstractpage}