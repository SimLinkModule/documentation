%!TEX root = Studienarbeit.tex

\documentclass[%
	pdftex,
	oneside,			% Einseitiger Druck.
	12pt,				% Schriftgroesse
	parskip=half,		% Halbe Zeile Abstand zwischen Absätzen.
	headheight = 12pt,	% Höhe der Kopfzeile
	headsepline,		% Linie nach Kopfzeile.
	footsepline,		% Linie vor Fusszeile.
	footheight = 16pt,	% Höhe der Fusszeile
	abstracton,		% Abstract Überschriften
	DIV=calc,		% Satzspiegel berechnen
	BCOR=8mm,		% Bindekorrektur links: 8mm
	headinclude=false,	% Kopfzeile nicht in den Satzspiegel einbeziehen
	footinclude=false,	% Fußzeile nicht in den Satzspiegel einbeziehen
	listof=totoc,		% Abbildungs-/ Tabellenverzeichnis im Inhaltsverzeichnis darstellen
	toc=bibliography,	% Literaturverzeichnis im Inhaltsverzeichnis darstellen
]{scrreprt}

% !TEX root =  arbeit.tex

\newcommand{\titel}{BLE HID Hardware-Erweiterungsmodul für Drohnenfernbedienungen}
\newcommand{\art}{Studienarbeit}
\newcommand{\autor}{Fabian Kuffer}
\newcommand{\studienbereich}{IT-Automotive}
\newcommand{\bearbeitungszeitraum}{4. Oktober 2022 - 8. Juni 2023}
\newcommand{\matrikelnr}{2044882}
\newcommand{\kurs}{TINF-20ITA}
\newcommand{\betreuer}{Prof. Dr. Karl Friedrich Gebhardt}
\newcommand{\datumAbgabe}{20. August 2022}
%!TEX root = Studienarbeit.tex

\usepackage{xstring}
\usepackage[utf8]{inputenc}
\usepackage[T1]{fontenc}
\usepackage{setspace}
\usepackage{longtable}

%bezeichnungen auf deutsch anpassen
\usepackage[english,ngerman]{babel}

\usepackage{makeidx}
\usepackage[margin=2.5cm,foot=1cm]{geometry}	% Seitenränder und Abstände
\usepackage[activate]{microtype} %Zeilenumbruch und mehr
%\usepackage[onehalfspacing]{setspace}
\usepackage[autostyle=true,german=quotes]{csquotes}
\usepackage{longtable}
\usepackage{enumitem}	% mehr Optionen bei Aufzählungen
\usepackage{graphicx}
\usepackage{pdfpages}   % zum Einbinden von PDFs
\usepackage{xcolor} 	% für HTML-Notation
\usepackage{float}
\usepackage{array}
\usepackage{calc}		% zum Rechnen (Bildtabelle in Deckblatt)
\usepackage[right]{eurosym}
\usepackage{wrapfig}
\usepackage{pgffor} % für automatische Kapiteldateieinbindung
\usepackage[perpage, hang, multiple, stable]{footmisc} % Fussnoten
\usepackage[printonlyused]{acronym} % falls gewünscht kann die Option footnote eingefügt werden, dann wird die Erklärung nicht inline sondern in einer Fußnote dargestellt
\usepackage{listings}

% Notizen. Einsatz mit \todo{Notiz} oder \todo[inline]{Notiz}. 
\usepackage[obeyFinal,backgroundcolor=yellow,linecolor=black]{todonotes}
% Alle Notizen ausblenden mit der Option "final" in \documentclass[...] oder durch das auskommentieren folgender Zeile
% \usepackage[disable]{todonotes}

% Literaturverweise
\usepackage[
	backend=biber,		% empfohlen. Falls biber Probleme macht: bibtex
	bibwarn=true,
	bibencoding=utf8,	% wenn .bib in utf8, sonst ascii
	sortlocale=de_DE,
	style=numeric,
]{biblatex}

% PDF Einstellungen
\usepackage[%
	pdftitle={\titel},
	pdfauthor={\autor},
	pdfsubject={\art},
	pdfcreator={pdflatex, LaTeX with KOMA-Script},
	pdfpagemode=UseOutlines, 		% Beim Oeffnen Inhaltsverzeichnis anzeigen
	pdfdisplaydoctitle=true, 		% Dokumenttitel statt Dateiname anzeigen.
	pdflang={de}, 			% Sprache des Dokuments.
]{hyperref}

% Workaround um Fehler in Hyperref, muss hier stehen bleiben
\usepackage{bookmark} %nur ein latex-Durchlauf für die Aktualisierung von Verzeichnissen nötig

%schriftart
\usepackage{lmodern}

%lorem ipsum generator
\usepackage{lipsum}

\addbibresource{quellen.bib}

% (Farb-)einstellungen für die Links im PDF
\definecolor{LinkColor}{HTML}{00007A}
\hypersetup{%
	colorlinks=true, 		% Aktivieren von farbigen Links im Dokument
	linkcolor=LinkColor, 	% Farbe festlegen
	citecolor=LinkColor,
	filecolor=LinkColor,
	menucolor=LinkColor,
	urlcolor=LinkColor,
	linktocpage=true, 		% Nicht der Text sondern die Seitenzahlen in Verzeichnissen klickbar
	bookmarksnumbered=true 	% Überschriftsnummerierung im PDF Inhalt anzeigen.
}

% Schriftart in Captions etwas kleiner
\addtokomafont{caption}{\small}


% Hurenkinder und Schusterjungen verhindern
% http://projekte.dante.de/DanteFAQ/Silbentrennung
\clubpenalty = 10000 % schließt Schusterjungen aus (Seitenumbruch nach der ersten Zeile eines neuen Absatzes)
\widowpenalty = 10000 % schließt Hurenkinder aus (die letzte Zeile eines Absatzes steht auf einer neuen Seite)
\displaywidowpenalty=10000

% Bildpfad
\graphicspath{{Bilder/}}

%für das Quellcodeverzeichnis
\renewcommand\lstlistingname{Quellcode}
\renewcommand\lstlistlistingname{Quellcodeverzeichnis}

%für die zählung von bildern
\counterwithout{figure}{chapter}
\counterwithout{table}{chapter}

%eine weitere ebene für titel
\newcommand{\subsubsubsection}[1]{\paragraph{#1}\mbox{}\\}