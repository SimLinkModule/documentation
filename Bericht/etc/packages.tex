%!TEX root = Studienarbeit.tex

\usepackage{xstring}
\usepackage[utf8]{inputenc}

\usepackage{cmbright}
\usepackage[T1]{fontenc}

\usepackage{setspace}

%bezeichnungen auf deutsch anpassen
\usepackage[english,ngerman]{babel}

\usepackage{makeidx}
\usepackage[margin=2.5cm,foot=1cm]{geometry}	% Seitenränder und Abstände
\usepackage[activate]{microtype} %Zeilenumbruch und mehr
%\usepackage[onehalfspacing]{setspace}
\usepackage[autostyle=true,german=quotes]{csquotes}
\usepackage{longtable}
\usepackage{enumitem}	% mehr Optionen bei Aufzählungen
\usepackage{graphicx}
\usepackage{pdfpages}   % zum Einbinden von PDFs
\usepackage{xcolor} 	% für HTML-Notation
\usepackage{float}
\usepackage{array}
\usepackage{calc}		% zum Rechnen (Bildtabelle in Deckblatt)
\usepackage[right]{eurosym}
\usepackage{wrapfig}
\usepackage{pgffor} % für automatische Kapiteldateieinbindung
\usepackage[perpage, hang, multiple, stable]{footmisc} % Fussnoten
\usepackage[printonlyused]{acronym} % falls gewünscht kann die Option footnote eingefügt werden, dann wird die Erklärung nicht inline sondern in einer Fußnote dargestellt
\usepackage{listings} %für quellcode

% Notizen. Einsatz mit \todo{Notiz} oder \todo[inline]{Notiz}. 
\usepackage[obeyFinal,backgroundcolor=yellow,linecolor=black]{todonotes}
% Alle Notizen ausblenden mit der Option "final" in \documentclass[...] oder durch das auskommentieren folgender Zeile
% \usepackage[disable]{todonotes}

% Literaturverweise
\usepackage[
	backend=biber,		% empfohlen. Falls biber Probleme macht: bibtex
	bibwarn=true,
	bibencoding=utf8,	% wenn .bib in utf8, sonst ascii
	sortlocale=de_DE,
	style=ieee,
]{biblatex}

% PDF Einstellungen
\usepackage[%
	pdftitle={\titel},
	pdfauthor={\autor},
	pdfsubject={\art},
	pdfcreator={pdflatex, LaTeX with KOMA-Script},
	pdfpagemode=UseOutlines, 		% Beim Oeffnen Inhaltsverzeichnis anzeigen
	pdfdisplaydoctitle=true, 		% Dokumenttitel statt Dateiname anzeigen.
	pdflang={de}, 			% Sprache des Dokuments.
]{hyperref}

% Workaround um Fehler in Hyperref, muss hier stehen bleiben
\usepackage{bookmark} %nur ein latex-Durchlauf für die Aktualisierung von Verzeichnissen nötig

%schriftart
\usepackage{lmodern}

%lorem ipsum generator
\usepackage{lipsum}

%für kontinuierliche nummerierung von bildern, tabellen und source code
\usepackage{chngcntr}

%benötigt für lange urls in literatur
\usepackage{url}

%verbinden mehrer Reihen einer Tabelle
\usepackage{multirow}