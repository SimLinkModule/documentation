%!TEX root = ../Studienarbeit.tex

\addchap{Abkürzungsverzeichnis}
%nur verwendete Akronyme werden letztlich im Abkürzungsverzeichnis des Dokuments angezeigt
%Verwendung: 
%		\ac{Abk.}   --> fügt die Abkürzung ein, beim ersten Aufruf wird zusätzlich automatisch die ausgeschriebene Version davor eingefügt bzw. in einer Fußnote (hierfür muss in header.tex \usepackage[printonlyused,footnote]{acronym} stehen) dargestellt
%		\acs{Abk.}   -->  fügt die Abkürzung ein
%		\acf{Abk.}   --> fügt die Abkürzung UND die Erklärung ein
%		\acl{Abk.}   --> fügt nur die Erklärung ein
%		\acp{Abk.}  --> gibt Plural aus (angefügtes 's'); das zusätzliche 'p' funktioniert auch bei obigen Befehlen
%	siehe auch: http://golatex.de/wiki/%5Cacronym
%	
\begin{acronym}[YTMMM]

    \acro{ATT}{Attribute Protocol}
    \acro{BBR}{Bluetooth Basic Rate}
    \acro{BLE}{Bluetooth Low Energy}
    \acro{CID}{Kanalidentifizierer}
    \acro{GAP}{Generic Access Profile}
    \acro{GATT}{Generic Attribute Pribute Profile}
    \acro{HCI}{Host Controller Interface}
    \acro{L2CAP}{Logical Link Control and Adaption Protocol}
    \acro{LL}{Link Layer}
    \acro{PHY}{Physical Layer}
    \acro{SDP}{Service Discovery Protocol}
    \acro{SIG}{Special Interest Group}
    \acro{SMP}{Security Manager Protocol}
    \acro{UUID}{universal unique identifier}
\end{acronym}