%!TEX root = ../Studienarbeit.tex

\addchap{Abkürzungsverzeichnis}
%nur verwendete Akronyme werden letztlich im Abkürzungsverzeichnis des Dokuments angezeigt
%Verwendung: 
%		\ac{Abk.}   --> fügt die Abkürzung ein, beim ersten Aufruf wird zusätzlich automatisch die ausgeschriebene Version davor eingefügt bzw. in einer Fußnote (hierfür muss in header.tex \usepackage[printonlyused,footnote]{acronym} stehen) dargestellt
%		\acs{Abk.}   -->  fügt die Abkürzung ein
%		\acf{Abk.}   --> fügt die Abkürzung UND die Erklärung ein
%		\acl{Abk.}   --> fügt nur die Erklärung ein
%		\acp{Abk.}  --> gibt Plural aus (angefügtes 's'); das zusätzliche 'p' funktioniert auch bei obigen Befehlen
%	siehe auch: http://golatex.de/wiki/%5Cacronym
%	
\begin{acronym}[YTMMM]

    \acro{ADC}{Analog-Digital-Wandler}
    \acro{API}{application programming interface}
    \acro{ATT}{Attribute Protocol}
    \acro{BBR}{Bluetooth Basic Rate}
    \acro{BLE}{Bluetooth Low Energy}
    \acro{CE}{conformit\'e europ\'eenne}
    \acro{CID}{Kanalidentifizierer}
    \acro{CPU}{central processing unit}
    \acro{CRC}{Cyclic redundancy check}
    \acro{ESD}{Electro-Static-Discharge}
    \acro{ESP-IDF}{Espressif IoT Development Framework}
    \acro{evdev}{Event Device}
    \acro{FCC}{Federal Communications Commission}
    \acro{FDM}{Fused Deposition Modeling}
    \acro{FIFO}{First In First Out}
    \acro{GAP}{Generic Access Profile}
    \acro{GATT}{Generic Attribute Profile}
    \acro{GPIO}{general purpose input/output}
    \acro{HCI}{Host Controller Interface}
    \acro{HID}{Human Interface Device}
    \acro{HOGP}{\acs{HID} over \acs{GATT} Profile}
    \acro{IC}{integrated circuit}
    \acro{IDE}{integrated development environment}
    \acro{IOCTL}{input output Control}
    \acro{ISM}{Industrial, Scientific and Medical}
    \acro{I2C}[I\textsuperscript{2}C]{inter integrated circuit}
    \acro{LDO}{Low-dropout regulator}
    \acro{LED}{light-emitting diode}
    \acro{LL}{Link Layer}
    \acro{L2CAP}{Logical Link Control and Adaption Protocol}
    \acro{MFi}{Made for iPod/iPhone/iPad}
    \acro{OLED}{organic light-emitting diode}
    \acro{PHY}{Physical Layer}
    \acro{PPM}{Puls-Positions-Modulation}
    \acro{RAM}{Random-access memory}
    \acro{RED}{Radio Equipment Directive}
    \acro{RPA}{Resolvable Private Address}
    \acro{SAR}{successive approximation}
    \acro{SDP}{Service Discovery Protocol}
    \acro{SIG}{Special Interest Group}
    \acro{SMP}{Security Manager Protocol}
    \acro{UART}{Universal Asynchronous Receiver Transmitter}
    \acro{USB}{Universal Serial Bus}
    \acro{UUID}{universal unique identifier}
    \acro{WLAN}{Wireless Local Area Network}
\end{acronym}