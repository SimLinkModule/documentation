%!TEX root = ../Studienarbeit.tex

\chapter{Einleitung}



Zwei arten von Drohnen. Freestyle/Renndrohen ... und Consumerdrohnen welche viele Sensoren haben und einfach zu fliegen sind.
Renndrohnen sind im Acro-Modus komplex zu fliegen, da dort viel gesteuert werden muss.
Es wird Training benötigt. Entweder im Freien oder in Simulatoren, um weniger zu zerstören.

Neben dem eigentlichen Multicopterfliegen stellt für Renn- und Freestyle-Multicopterpiloten das Training einen wichtigen Bestandteil dar. Dieses kann in zwei Varianten durchgeführt werden. Der
Multicopterpilot trainiert entweder am Flugplatz. Hier können aber durch Abstürze hohe Reparaturkosten und lange Reparaturzeiten entstehen. Oder der Multicoperpilot trainiert im Simulator am Rechner.
Damit die gewohnte Fernbedienung ebenfalls am Rechner verwendet werden kann, bieten einige Hersteller die Möglichkeit an, die Fernbedienung als USB-HID-Joystick zu verwenden.
Durch die immer leistungsfähiger werdenden Smartphones und Tablets wäre es wünschenswert, auf mobilen Geräten Simulatoren für das Training zu verwenden. Das Problem hierbei ist jedoch, dass die
Verbindung der Multicopterfernbedienung mit dem mobilen Gerät über USB nur eingeschränkt beziehungsweise unmöglich ist. Beseitigt werden kann dieses Problem bei einigen Fernbedienungen mit
Modulschächten, mit Hilfe derer die Tasten- und Joysticksignale über andere Funkstandards übertragen werden können.
Ziel der Arbeit ist es, ein Hardware-Erweiterungsmodul für Multicopterfernbedienungen zu entwickeln. Vorausgesetzt wird im Rahmen dieser Arbeit, dass die Fernbedienungen einen Modulschacht aufweisen
und die Firmware OpenTX beziehungsweise eine Abspaltung davon verfügbar ist. Das Erweiterungsmodul soll sich dabei durch BLE als HID-Gerät an Endgeräten authentifizieren, wodurch die
Multicopterfernbedienung als kabelloser Joystick an Endgeräten verwendet werden kann.
Weitere zusätzliche Optionen -- sofern zeitlich machbar -- sind zum einen, ein kleines LED-Display einzubauen, womit die Bedienung des Moduls erleichtert werden kann. Zum anderen eine GUI zu entwickeln,
um den Updateprozess für das BLE-HID-Modul zu vereinfachen. Die GUI kann dafür mit dem Framework Electron für eine systemunabhängige Verwendung entwickelt werden.

\section{Motivation}

In den letzten Jahren ist die Leistungsfähigkeit von Tablets gestiegen. Jedoch ist es schwer mittels USB eine Verbindung aufzubauen. Dafür muss auf ein Funkstandard ausgewichen werden --> Bluetooth.
Es wird dann genau BLE verwendet, da dieses unter Apple ohne Einschränkungen verwendet werden kann.
Dadurch findet eine Ausweitung für Simulatoren auf mobile Geräte statt, da es zurzeit die Kommunikation mit Tablets schwer ist.
--> Schreiben, dass es mittels einem Modul an Controllern gelöst werden soll.

\section{Stand der Technik}

Gibt im Umfeld nur wenig bis keine BLE HID Geräte zum Verbinden mit Smartphone. Eine Möglichkeit via USB und via Betaflight-Flightcontroller.

Schreiben, was es für andere Module statt ESP gibt. Der Modulschacht wird zurzeit nur für andere Übertragungsstandards für Drohnen verwendet.