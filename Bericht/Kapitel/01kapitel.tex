%!TEX root = ../Studienarbeit.tex

\chapter{Einleitung}
Enthält Problemstellung, Ziel und Vorgehensweise der Arbeit (Gegenstand und Ziele der Arbeit/Aufgabenbeschreibung,
geplante Vorgehensweise, Einführung in Thema, Motivation der Aufgabenstellung/Vorausblick)

Grundlagen (z.B. Stand der Technik/Forschung)

Hauptteil (Anforderungsdefinition, Anforderungsanalyse, Lösungsgenerierung, Lösungsbewertung, Umsetzung),
ggf. in mehreren sinnvollen Gliederungspunkten

Kritische Reflexion und Ausblick
------

Erste Erwähnung eines Akronyms wird als Fußnote angezeigt. Jede weitere wird
nur verlinkt: \acf{AGPL}. \cite{fsf, baumgartner}

Nur erwähnte Literaturverweise werden auch im Literaturverzeichnis gedruckt:
\cites[150--130]{baumgartner}[S.~150f.]{dreyfus}

\lipsum[1]

\section{lorem ipsum}

\subsection{asdf}
\subsubsection{jklö}