%!TEX root = ../Studienarbeit.tex

\chapter{Aufgabenstellung}
\label{section:Aufgabenstellung}

Ziel der Arbeit ist es, ein Hardware-Erweiterungsmodul für Multikopterfernsteuerungen zu entwickeln. Vorausgesetzt wird im Rahmen dieser Arbeit, dass die Fernsteuerungen einen Modulschacht aufweisen und die Firmware OpenTX \cite{opentxMain} beziehungsweise eine Abspaltung davon verfügbar ist. Das Erweiterungsmodul soll sich dabei durch \acs{BLE} als \acs{HID}-Gerät an Endgeräten authentifizieren, wodurch die Multikopterfernsteuerung als kabelloser Joystick an Endgeräten verwendet werden kann. Die Umsetzung dieser Arbeit lässt sich in nachfolgende drei Teilbereiche aufteilen.

\section{Softwareentwicklung}
\label{section:softwareRequirement}
Im Aufgabenbereich der Softwareentwicklung soll die Kommunikation zwischen dem ESP32-Entwicklerboard und Windows, Linux, iOS/ iPadOS und Android-Systemen hergestellt werden. MacOS soll kein Teil der unterstützen Betriebssysteme sein, da kein Endgerät mit MacOS zum Testen vorhanden ist. Die Kommunikation soll dabei mittels \acs{BLE} stattfinden und das Entwicklerboard soll sich als \acs{HID}-Gerät authentifizieren. Des Weiteren soll die Kommunikation zwischen dem ESP32-Entwicklerboard und der Fernsteuerung mittels des Modulschachts der Fernsteuerung implementiert werden. Dafür soll auf eines der vorhandenen Protokolle der Fernsteuerung zurückgegriffen werden, damit die Firmware der Fernsteuerung nicht angepasst werden muss. Der letzte Bestandteil dieses Aufgabenbereichs ist die Implementierung weiterer Möglichkeiten der Eingabe und Ausgabe an dem ESP32-Entwicklerboard. Dafür soll zum einen ein 0,91~Zoll großes \acs{OLED}-Display verwendet werden, um kurze Statusnachrichten anzuzeigen. Ebenso sollen Status-\acs{LED}s die Interaktion mit dem Modul vereinfachen. Die Bestimmung des Akkustandes der Fernsteuerung soll mittels des integrierten \acp{ADC} des ESP32-Entwicklerboard geschehen, da der Akkustand bei \acs{HID}-Geräten bereitgestellt werden muss.

\section{Platinenentwurf}

In diesem Aufgabenbereich soll der erstellte Steckbrettaufbau, der während der Softwareentwicklung benötigt wurde, in eine Platine umgewandelt werden. Ein Bestandteil dieser Platine soll das ESP32-Modul sein, welches als primärer Mikrocontroller fungiert. Ebenso soll eine Spannungsregulierung für die Komponenten der Platine erstellt werden, da die Elektronik über die Stromversorgung der Fernsteuerung betrieben werden soll. Ein weiterer Bestandteil der Platine ist die Verbindung der Ein- und Ausgabeelemente mit dem ESP32-Modul, um die Bedienung des Erweiterungsmoduls zu vereinfachen. Zur Umsetzung soll auf bereits vorhandene Referenzdesigns des ESP32-Entwicklerboards zurückgegriffen werden.

\section{Gehäuseerstellung}

Im letzten Aufgabenbereich soll ein Gehäuse für die erstellte Platine hergestellt werden, damit die Platine in den Modulschacht vom Typ Lite fest verbaut werden kann. Ebenso muss bei der Konstruktion beachtet werden, dass das Gehäuse möglichst ohne Stützstrukturen mittels eines 3D-Druckers gedruckt werden kann. Dadurch soll die Nachbearbeitung des Gehäuses nach dem Druck auf ein Minimum reduziert werden.