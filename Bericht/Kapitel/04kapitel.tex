%!TEX root = ../Studienarbeit.tex

\chapter{Umsetzung}

\section{Softwareentwicklung}
In diesen Abschnitt werden einige Kernkomponenten der Software detaillierter dargestellt und die Auswahl des Mikrocontrollers beschrieben.

\subsection{Auswahl der Mikrocontrollers}
Als Mikrocontroller wird das ESP32-WROOM-32E-Modul verwendet. Dies hat mehrere Gründe, welche nachfolgend zu finden sind.
Als erster Grund, welcher für die Verwendung des ESP32 spricht, ist die große Beliebtheit des Mikrocontrollers in Hobbyprojekten und daraus folgernd eine große Community die bei Problemen in der Entwicklung unterstützen kann \cites{redditESP32}{esp32Forum}. Ebenfalls ist eine gute Dokumentation der \ac{ESP-IDF} durch Kommentare im Sourcecode, sowie durch Beispielprogramme gegeben. Als weiterer Grund für die Verwendung des ESP32 spricht, die Unterstützung mehrerer Bluetooth-Stacks, welche in Kapitel \ref{section:bluetoothStacks} beschrieben sind. Als letzter Grund ist noch die Verfügbarkeit der Mikrocontroller in kompakten Hardwaremodulen, in denen alle benötigen Hauptkomponenten für den Betrieb des Mikrocontrollers enthalten sind, wie Kondensatoren, Widerständen und in manchen Modulen ebenso die benötigte Antenne für den Betrieb von Bluetooth oder \acs{WLAN} \cite[S.~14]{espressifHardwareDesignGuidelines}. Auch haben die ESP32-Module meist eine CE-Zertifizierung, wie in Kapitel \ref{section:esp32Explained} beschrieben ist und dadurch könnte das entwickelte Fernsteuerungsmodul vereinfacht für den Vertieb zertifiziert werden. Trotz den vorhandenen Funktionsumfang und Ökosystem ist das ESP32-Modul kostengünstig und im Bereich um 5~€ zu erwerben \cite{espressifModules}.

\subsection{Auswahl des Bluetooth-Stacks}
Als Bluetooth-Stack für den ESP32 wird der Open Source \ac{BLE}-Stack Apache NimBLE verwendet. Dies hat den Hintergrund, dass die Kommunikation ausschließlich zwischen dem Fernsteuerungserweiterungsmodul und Endgeräten mittels \ac{BLE} erfolgen soll. Für diesen Einsatzzweck wird die Verwendung von Apache NimbLE empfholen, da dieser kompakter in der Codegröße ist und weniger Speicher zur laufzeit benötigt \cite{espidfBluetoothStack}.

\subsection{Kommunikation zwischen dem Mikrocontroller und dem Endgerät}
%Schreiben warum hid profil, da dafür kein Treiber geschrieben werden muss und ebenso die Zertifizierung durch Endgeräthersteller entfällt wie beispielsweise bei Apple.
%verwendete services und charakteristiks aufschreiben
%clients können sich subscriben damit diese neue Daten des Moduls automatisch bekommen
%finale Hid Datenstrukur anzeigen --> 8 analoge Kanäle und 8 digitale Kanäle --> Wertebereich musste angepasst werden
%rpa bei bluetooth erklären
%unterschied zwischen bonding und pairing ble
%verschiedene maßnahmen für encrypting aufschreiben und ausgewählte aufschreiben
%GAP events erklären
%auch schreiben, dass die pflichangaben für BLE unter apple gefolgt wurden
\subsection{Kommunikationsprotokoll zwischen der Multikopterfernsteuerungen und dem Mikrocontroller}
%schreiben warum crsf gewählt
%schreiben wie es umgesetzt wurde auf dem ESP mittels dem driver und wie der driver funktioniert
%Wie die Datenstruktur ausgelesen wird --> die ersten 3 bit überprüft und die länge
%crc prüfung hinzugefügt
%Daten werden in einen globalen Strukt gelagert --> struct aufzeigen
%Daten werden ausgewertet wenn der UART Buffer voll ist oder ein gewisser Timeout zwischen gesendet daten stattgefunden hat
\subsection{Statusausgabe des Mikrocontrollers mittels eines OLED-Displays}
%Display erklären und wie dort geschrieben werden kann.
%I2C verwendet

\subsection{Kombination aller Softwarekomponenten}
Als Programmiersprache wird C verwendet
%Datenaustausch mittels globale Variablen
%Aufgabenausführung am ESP erklären
    %--> Display wird im Code ausgeführt
    %--> Buttonklick via Interrupts
    %--> BLE Task
    %--> Ermittlung des Spannungslevels mittels ADC und Timer
    %--> Auslesen des CRSF Task



\section{Platinenentwurf}
%PCB-Design erklären. (Erklären für was die zonen sind und was beachtet werden musste, batterie auslesen, esd schutz, usb zu serial, schutzschaltung strom, Spannungsregulierung Datenleitungen)
%sektionen des PCB erklären
%schreiben das sich an die Referenzdesing von folgenden Quellen gehalten wurde: Liste von Quellen finden
% Im Anhag die Schematic aller PCBs einfügen.

\section{Gehäuseerstellung}
%besonderheiten des Gehäuses aufzeigen --> im bezug auf 3d druck
%3d cad dateien des gehäuses einfügen