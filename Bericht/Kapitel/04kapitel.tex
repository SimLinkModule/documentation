%!TEX root = ../Studienarbeit.tex

\chapter{Umsetzung}

\section{Softwareentwicklung}

\section{Auswahl der Mikrocontrollers}
%Das ESP32-WROOM-32-Modul soll für die BLE-Kommunikation verwendet werden, da es ein kostengünstiges und nach CE zertifiziertes Modul ist.

\subsection{Auswahl des Bluetooth-Stacks}
%Empfohlen wird für BLE-only andwendung NimBLE zu verwenden, da dieser eine größere Codegröße hat und weniger Speicher zur laufzeit benötigt. Weshalb für diese Arbeit NimBLE verwendet wird. \cite{espidfBluetoothStack}

\subsection{Kommunikationsprotokoll zwischen dem Mikrocontroller und dem Endgerät}
%Schreiben warum hid profil, da dafür kein Treiber geschrieben werden muss und ebenso die Zertifizierung durch Endgeräthersteller entfällt wie beispielsweise bei Apple.
%verwendete services und charakteristiks aufschreiben
%clients können sich subscriben damit diese neue Daten des Moduls automatisch bekommen
%finale Hid Datenstrukur anzeigen --> 8 analoge Kanäle und 8 digitale Kanäle --> Wertebereich musste angepasst werden
%rpa bei bluetooth erklären
%unterschied zwischen bonding und pairing ble
%verschiedene maßnahmen für encrypting aufschreiben und ausgewählte aufschreiben
%GAP events erklären
%auch schreiben, dass die pflichangaben für BLE unter apple gefolgt wurden
\subsection{Kommunikationsprotokoll zwischen der Multikopterfernsteuerungen und dem Mikrocontroller}
%schreiben warum crsf gewählt
%schreiben wie es umgesetzt wurde auf dem ESP mittels dem driver und wie der driver funktioniert
%Wie die Datenstruktur ausgelesen wird --> die ersten 3 bit überprüft und die länge
%crc prüfung hinzugefügt
%Daten werden in einen globalen Strukt gelagert --> struct aufzeigen
%Daten werden ausgewertet wenn der UART Buffer voll ist oder ein gewisser Timeout zwischen gesendet daten stattgefunden hat
\subsection{Statusausgabe des Mikrocontrollers mittels eines OLED-Displays}
%Display erklären und wie dort geschrieben werden kann.
%I2C verwendet

\subsection{Kombination aller Softwarekomponenten}
%Datenaustausch mittels globale Variablen
%Aufgabenausführung am ESP erklären
    %--> Display wird im Code ausgeführt
    %--> Buttonklick via Interrupts
    %--> BLE Task
    %--> Ermittlung des Spannungslevels mittels ADC und Timer
    %--> Auslesen des CRSF Task



\section{Platinenentwurf}
%PCB-Design erklären. (Erklären für was die zonen sind und was beachtet werden musste, batterie auslesen, esd schutz, usb zu serial, schutzschaltung strom, Spannungsregulierung Datenleitungen)
%sektionen des PCB erklären
%schreiben das sich an die Referenzdesing von folgenden Quellen gehalten wurde: Liste von Quellen finden

\section{Gehäuseerstellung}
%besonderheiten des Gehäuses aufzeigen --> im bezug auf 3d druck
%3d cad datein des gehäuses einfügen