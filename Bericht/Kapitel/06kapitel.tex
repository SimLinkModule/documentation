%!TEX root = ../Studienarbeit.tex

\chapter{Rekapitulation und Ausblick}
\section{Rekapitulation}

Die für diese Arbeit notwendigen Aufgaben wurden in vier Bereiche aufgeteilt. Im ersten Teilbereich der Softwareentwicklung wurde die nötige Kommunikation zwischen der Multikopterfernsteuerung und dem Erweiterungsmodul implementiert. Als Kommunikationsprotokoll wurde CRSF verwendet, da dies die höchste Übertragungsrate bei möglichst kleinen Datenpaketen, von allen vorhandenen Kommunikationsprotokollen, bietet. Ebenso wurde die Kommunikation zwischen dem Erweiterungsmodul und Endgeräten entwickelt. Als Kommunikationsprotokoll wurde hierfür \ac{BLE} verwendet, indem die Fernsteuerungsdaten mittels \ac{HOGP} verpackt sind. Damit ein Endanwender mit dem Erweiterungsmodul interagieren kann, wurden zusätzlich Eingabemöglichkeiten mittels Tastern hinzugefügt. Für die Ausgabe von Statusinformationen wird ein \ac{OLED}-Display verwendet, welches zum Datenaustausch \ac{I2C} verwendet. Der zweite Teilbereich befasste sich mit dem Entwurf von Platinen für das Erweiterungsmodul. Dies war nötig damit das Erweiterungsmodul kompakt, mobil und benutzerfreundlich verwendet werden kann. Dafür wurden final drei Platinen erstellt, welche alle Elektronikkomponenten enthalten, wie beispielsweise den ESP32-Mikrocontroller, das \ac{OLED}-Display, mehrere Eingabetaster, einer Spannungsregulierung und eine Buchse für die Verbindung mit der Multikopterfernsteuerung. Der dritte Teilbereich umfasste die Entwicklung eines Gehäuses, um die entworfenen Platinen fest und kompakt im Modulschacht (Typ: Lite) der Fernsteuerung befestigt werden kann. Hierbei wurde darauf geachtet, dass das Gehäuse mittels eines \ac{FDM}-3D-Druckers gedruckt werden kann und eine möglichst geringe Nachbearbeitung benötigt. Der letzte Teilbereich befasste sich mit der Validierung des Funktionsumfangs und dem Latenzvergleich zwischen dem Erweiterungsmodul und der internen USB-Schnittstelle der Multikopterfernsteuerung. In der Validierung konnte festgestellt werden, das der größte Teil der ursprünglichen Anforderungen umgesetzt worden konnte. Jedoch konnte eine größere Funktion nicht umgesetzt werden, welche die Verwendung des Erweiterungsmoduls unter iOS- und iPadOS-Geräten ist. Rein technisch konnte diese Funktion umgesetzt werden, was durch die internen \ac{HCI}-Daten des Endgeräts nachgewiesen werden kann. Es stellte sich heraus, dass das Erweiterungsmodul zusätzlich durch das \ac{MFi}-Programm von Apple zertifiziert werden müsste, was einen großen Kosten- und Arbeitsaufwand mit sich bringt. Im Latenzvergleich zwischen der internen USB-Verbindung der Fernsteuerung und der \ac{BLE}-Verbindung des Erweiterungsmoduls, konnte festgestellt werden, dass die Latenz der \ac{BLE}-Verbindung im Durchschnitt 20,84~ms größer ist (Optimalfall). Durchschnittlich beträgt die Latenz der \ac{BLE}-Verbindung 30~ms und hat dabei eine Standardabweichung von 7,68~ms im Optimalfall. Zugehörig ist in Kapitel \ref{section:resultExplanation} beschrieben, wie der Optimalfall der \ac{BLE}-Verbindung zu Stande kommt.

Anzumerken ist, dass die Latenz zwischen Multikopterfernsteuerung und einem Endgerät mittels des \ac{BLE}-Erweiterungsmoduls, im Vergleich zu der vorhandenen USB-Verbindung, angestiegen ist. Auch werden nur die nötigsten \ac{GAP}-Events durch das Erweiterungsmodul ausgewertet, wodurch hilfreiche Informationen vor dem Endnutzer verschwiegen werden. Zuletzt ist noch anzumerken, dass die Produktionskosten des Erweiterungsmoduls für die Eigenverwendung zu hoch ausfallen (ungefähr 180~€). Alternativ zur Produktion der Erweiterungsplatinen und dem 3D-Druck eines Gehäuse, kann auch ein ESP32-Entwicklungsboard und wenigen weiteren Elektronikkomponenten ebenso das Erweiterungsmodul nachgebaut werden. Hierdurch können die Kosten auf ungefähr 50~€ gesenkt werden. 

Zusammenfassend ist mit dem entwickelten Erweiterungsmodul es nun möglich Multikopterfernsteuerungen modellunabhängig mit einem Endgerät mittels \ac{BLE} zu verbinden. Einschränkungen hierbei sind, dass die Multikopterfernsteuerung das CRSF-Protokoll für die Übertragung am Modulschacht unterstützt und dass das Betriebssystem des Endgeräts Windows, Linux oder Android ist. Außerdem ist die Software, der Platinenentwurf sowie das Gehäusedesign unter der open source Lizenz \textit{Apache-2.0} öffentlich zugänglich bereitgestellt.

\section{Ausblick}
Aufbauend auf den Ergebnissen dieser Arbeit, können weitere Anpassungen in den Bereichen Platinenentwurf, Gehäuseentwurf und Softwareentwicklung durchgeführt werden.

Im Bereich des Platinenentwurfs können Anpassungen erfolgen, um die Platinen kleiner und kompakter gestalten zu können. Dafür kann zum einen die Verbindungsplatine zwischen der Multikopterfernsteuerung und dem Erweiterungsmodul in die Hauptplatine des Erweiterungsmoduls integriert werden. Zum anderen kann die Logik für die Programmierung des ESP32-Mikrocontroller entfallen, da diese Logik nur selten benötigt wird und der Start des Mikrocontrollers in den Programmiermodus auch manuell durchgeführt werden kann (siehe Kapitel \ref{section:pcbImplementation}). Zusätzlich kann die Platinengröße verringert werden, indem die benötigten Elektronikkomponenten beidseitig auf den Platinen angebracht werden.

Im Bereich des Gehäuseentwurfs kann einerseits ein Adapter konstruiert werden, um das Erweiterungsmodul (Typ: Lite) für normale Modulschächte vorzubereiten. Zusätzlich wird hierfür eine Adapterplatine benötigt, welche das Buchsenlayout zwischen den verschiedenen Modulschachtarten wandelt. Andererseits können durch die erwähnten Platinenanpassungen das Gehäuse ebenso kompakter gestaltet werden, wodurch das \acs{OLED}-Display für eine bessere Lesbarkeit näher an der Gehäuseaußenkante angebracht werden kann.

Im Bereich der Softwareentwicklung kann die Auswertung der verfügbaren \ac{GAP}-Events implementiert werden, um Benutzern des Erweiterungsmoduls weiterführende Informationen bereitzustellen. Ebenso können weitere Kommunikationsprotokolle zwischen der Multikopterfernsteuerung und dem Erweiterungsmodul implementiert werden, um weitere Fernsteuerungsmodelle zu unterstützen. Die Implementierung von digitalen Kommunikationsprotokolle ist ohne Hardwareanpassungen möglich. Jedoch muss für die Implementierung von analogen Kommunikationsprotokollen zusätzlich der analoge Pin des Modulschachts mit dem ESP32-Mikrocontroller verbunden werden.