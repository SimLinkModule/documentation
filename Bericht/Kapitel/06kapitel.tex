%!TEX root = ../Studienarbeit.tex

\chapter{Rekapitulation und Ausblick}
\todo[inline]{TODO: Rekapitulation und Ausblick}
%Rekapitulation
%-Kurze zusammenfassung wie Arbeit in Teilbereiche aufgebaut wurde und was die ergebnisse davon sind
%-Verbesserungen anbringen / Nachteile
%-Zusammenfassen, was jetzt möglich ist nach dieser Arbeit mit ergebnissen

%Ausblick:
%-Platine könnte kompakter hergestellt werden --> Indem die Connector- und main-platine kombiniert wird und auf der mainplatine der teil für den vereinfachten Flash weggelassen wird
%-Gehäuse kann niedriger hergestellt werden, wenn die Platine angepasst wird und der Stecker für die IO-PCB nicht nach oben wegsteht
%-Display kann näher an das Gehäuse, wenn die Platine statt nur einseitig, beidseitg bestückt wird
%-Es können weiter Kommunikationsprotokolle implementiert werden --> nur digitale für Analog müsste der Pin mit Mikrocontroller noch verbunden werden
%-Es kann auch noch ein Adapter gebaut werden womit das light modul in einen normalen modulschaft verwendet werden kann
%-Ausführlichere Hinweise können am Display angezeigt werden, da es noch mehr GAP-Events gibt welche aktuell nicht verwendet werden