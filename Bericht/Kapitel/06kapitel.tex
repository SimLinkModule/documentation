%!TEX root = ../Studienarbeit.tex

\chapter{Rekapitulation und Ausblick}
\todo[inline]{TODO: Rekapitulation und Ausblick}
%Rekapitulation
%-Kurze zusammenfassung wie Arbeit in Teilbereiche aufgebaut wurde und was die ergebnisse davon sind
    %-Teilbereiche: Softwareentwicklung, Gehäusedesign, Platinendesign und Validierung der Funktionen und Latenzauswertung
    %-Softwareentwicklung: CRSF-Daten werden ausgewertet und mittels BLE an Endgerät verschickt durch ESP32. Zusätzliche Statusmitteilungen und eingabemöglichkeiten durch Buttons und OLED-Display
    %-Gehäusedesign: Gehäuse mit geringer nachbearbeitung druckbar; Verwendbar für Lite-Modulschächte
    %-Platinendesign: Platine in denen alle wichtigen Elektronikkomponenten für das Erweiterungsmodul enthalten ist; Mit zusätzlicher Logik für das einfache Programmieren des ESP32
    %-Funktionsumfang: Größter Teil der ursprünglichen Anforderungen konnten umgesetzt werden. Größte Funktion die nicht umgesetzt worden konnte ist die Verwendung unter iOS-Geräten, da das Erweiterungsmodul in das Mfi programm aufgenommen werden müsste, was zustätzliche Kosten sind
    %-Latenzauswertung die Latenz weißt im optimalfall eine Latenz von 30ms auf mit einer Standardabweichung von 7,68ms. Der optimalfall ist in Kaptitel ... beschrieben.
%-Verbesserungen anbringen / Nachteile
    %-gestiegene Latenz
    %-Nur die nötigsten GAP-Evente ausgewertet
    %-Für die eigenbenutzung ist die Platinen herstellung und bestückung zu teuer ungefähr 180$, jedoch kann man es auch in einer minimalen Version auch nur mit einen Entwicklungsboard verwenden die Software und einer Modulsteckerbuchse.
%-Zusammenfassen, was jetzt möglich ist nach dieser Arbeit mit ergebnissen
    %-Fernsteuerungsunabhängig, solange das CRSF-Protokoll über den Modulschacht übetragen wird
    %-Open Source MIT Lizenz vorhandenes Erweiterungsmodul
    %-BLE-HID-Controller für Simulatoren unter Linux, Windows und Android

%Ausblick:
%-Platine könnte kompakter hergestellt werden --> Indem die Connector- und main-platine kombiniert wird und auf der mainplatine der teil für den vereinfachten Flash weggelassen wird
%-Gehäuse kann niedriger hergestellt werden, wenn die Platine angepasst wird und der Stecker für die IO-PCB nicht nach oben wegsteht
%-Display kann näher an das Gehäuse, wenn die Platine statt nur einseitig, beidseitg bestückt wird
%-Es können weiter Kommunikationsprotokolle implementiert werden --> nur digitale für Analog müsste der Pin mit Mikrocontroller noch verbunden werden
%-Es kann auch noch ein Adapter gebaut werden womit das light modul in einen normalen modulschaft verwendet werden kann
%-Ausführlichere Hinweise können am Display angezeigt werden, da es noch mehr GAP-Events gibt welche aktuell nicht verwendet werden