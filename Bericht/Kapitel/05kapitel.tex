%!TEX root = ../Studienarbeit.tex

\chapter{Validierung und Gegenüberstellung}

\section{Validierung des Funktionsumfangs}

In diesen Abschnitt der Studienarbeit soll der definierte Aufgabenumfang von Kapitel \ref{section:Aufgabenstellung} mit dem realisierten Umfang verglichen werden und gegebenenfalls vorhandende Einschränkungen erläutert werden.

\subsection{Softwareentwicklung}
%ios problem, dass es nicht geht --> Trotz der Aussagen auf der Website von Apple Mfi und dem Vergleich mit einen XBox Controller, welcher per BLE mit iOS verbunden wird und funktioniert konnte keine Verbindung hergestellt werden. Dies ist auch unter folgender Quelle zu sehen, dass es nicht wie auf der Website klingt zu machen ist. https://github.com/lemmingDev/ESP32-BLE-Gamepad/issues/50#issuecomment-862770124 --> iOS verbindet sich mit dem Modul und submitted zu den HID daten aber es werden die Daten nicht weitergegeben.
%auch schreiben, dass das orqa modul gekauft wurde, da es vermeindlich auf unter ios kompatibel ist --> funktioniert auch nicht unter ios bzw. es wurde auch nicht als controller erkannt
%auch schreiben, dass mit xbox controller und switch controller die HCI daten auf einen idevice verglichen worden sind. Sowohl beim Pairing, erneuter Verbindungsaufbau und auch Advertising Pakete
% Switch Controller versendet HID-Daten mittels Bluetooth Classic --> konnte nicht für den Vergleich verwendet werden

%schreiben das es mit einer Fernsteuerung auf dem eine Abspaltung von OpenTX vorhanden ist funktioniert und ein Modulschacht hat
%Die Kommunikation mittels BLE funktioniert auch; HID authentifizierung auch und es wird ein integriertes Protokoll von OPENTX verwendet
%OLED Display für Status ausgaben vorhanden und Knöpfe für die Steuerung des Moduls

%Battery anzeige konnte mittels dem ADC nicht verwendet werden, da es keinen PIn an der Fernsteuerung gibt, wie angenommen bei Lite Modulschacht bei normalen Schacht schon, deswegen wird Batteriespannung immer auf 0% gesetzt

%es musst in der Software keine LED ansteuerung stattfinden, da alle Statusmeldungen eindeutig über das Display angezeigt werden konnte. Es wurden nur eine LED verwendet um anzuzeigen, dass ein Stromfluss vorhanden ist in der Schaltung (rein in hardware).

%Schauen ob das Gerät unter Linux, Android, Windows, iOS funktioniert. Schauen ob die Kommunikation mit dem Kontroller funktioniert. --> Wo funktioniert als kabelloser Joystick

\subsection{Platinenentwurf}
%Platine enthält Mikrocontroller und spannungsregulierung, Button und Display sowie LED für die Ein und Ausgabe vorhanden

\subsection{Gehäuseerstellung}
%Gehäuse passt in Lite-Schacht; Gehäuse kann mit minimaler Anzahl von Stützstrukturen gedruckt werden --> Nur 4 Stück die ohne Aufwand entfernt werden können und keine Nachbearbeitung nach dem 3D-Druck erfordern

\section{Gegenüberstellung BLE-Modul und USB-Verbindung}
%Test zunächst mit servo probiert, um nicht an Platine direkt arbeiten zu müssen. Jedoch ist der Delay nicht in einen glaubwürdigen bereich, da zu lang.
%Eine Studie gefunden, bei dem optokoppler verwendet wurden, und verschiedene Geräte getestet wurde als vergleichswert verwendbar.
%Dadurch neuer Versuchsaufbau mit optokoppler. Schauen ob es in einen bereich mit den restlichen ist und wie viel schlechter es wurde.
%Vielleicht gaußverteilung darstellen und werte dafür raussrechnen. Schreiben dass x mal getestet wurde.