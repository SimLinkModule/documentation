%!TEX root = Studienarbeit.tex

\documentclass[%
	pdftex,
	oneside,			% Einseitiger Druck.
	12pt,				% Schriftgroesse
	parskip=half,		% Halbe Zeile Abstand zwischen Absätzen.
	headheight = 12pt,	% Höhe der Kopfzeile
	headsepline,		% Linie nach Kopfzeile.
	footsepline,		% Linie vor Fusszeile.
	footheight = 16pt,	% Höhe der Fusszeile
	abstracton,		% Abstract Überschriften
	DIV=calc,		% Satzspiegel berechnen
	BCOR=8mm,		% Bindekorrektur links: 8mm
	headinclude=false,	% Kopfzeile nicht in den Satzspiegel einbeziehen
	footinclude=false,	% Fußzeile nicht in den Satzspiegel einbeziehen
	listof=totoc,		% Abbildungs-/ Tabellenverzeichnis im Inhaltsverzeichnis darstellen
	toc=bibliography,	% Literaturverzeichnis im Inhaltsverzeichnis darstellen
]{scrreprt}

% !TEX root =  arbeit.tex

\newcommand{\titel}{BLE HID Hardware-Erweiterungsmodul für Drohnenfernbedienungen}
\newcommand{\art}{Studienarbeit}
\newcommand{\autor}{Fabian Kuffer}
\newcommand{\studienbereich}{IT-Automotive}
\newcommand{\bearbeitungszeitraum}{4. Oktober 2022 - 8. Juni 2023}
\newcommand{\matrikelnr}{2044882}
\newcommand{\kurs}{TINF-20ITA}
\newcommand{\betreuer}{Prof. Dr. Karl Friedrich Gebhardt}
\newcommand{\datumAbgabe}{20. August 2022}
%!TEX root = Studienarbeit.tex

\usepackage{xstring}
\usepackage[utf8]{inputenc}

\usepackage{cmbright}
\usepackage[T1]{fontenc}

\usepackage{setspace}
\usepackage{longtable}

%bezeichnungen auf deutsch anpassen
\usepackage[english,ngerman]{babel}

\usepackage{makeidx}
\usepackage[margin=2.5cm,foot=1cm]{geometry}	% Seitenränder und Abstände
\usepackage[activate]{microtype} %Zeilenumbruch und mehr
%\usepackage[onehalfspacing]{setspace}
\usepackage[autostyle=true,german=quotes]{csquotes}
\usepackage{longtable}
\usepackage{enumitem}	% mehr Optionen bei Aufzählungen
\usepackage{graphicx}
\usepackage{pdfpages}   % zum Einbinden von PDFs
\usepackage{xcolor} 	% für HTML-Notation
\usepackage{float}
\usepackage{array}
\usepackage{calc}		% zum Rechnen (Bildtabelle in Deckblatt)
\usepackage[right]{eurosym}
\usepackage{wrapfig}
\usepackage{pgffor} % für automatische Kapiteldateieinbindung
\usepackage[perpage, hang, multiple, stable]{footmisc} % Fussnoten
\usepackage[printonlyused]{acronym} % falls gewünscht kann die Option footnote eingefügt werden, dann wird die Erklärung nicht inline sondern in einer Fußnote dargestellt
\usepackage{listings}

% Notizen. Einsatz mit \todo{Notiz} oder \todo[inline]{Notiz}. 
\usepackage[obeyFinal,backgroundcolor=yellow,linecolor=black]{todonotes}
% Alle Notizen ausblenden mit der Option "final" in \documentclass[...] oder durch das auskommentieren folgender Zeile
% \usepackage[disable]{todonotes}

% Literaturverweise
\usepackage[
	backend=biber,		% empfohlen. Falls biber Probleme macht: bibtex
	bibwarn=true,
	bibencoding=utf8,	% wenn .bib in utf8, sonst ascii
	sortlocale=de_DE,
	style=ieee,
]{biblatex}

% PDF Einstellungen
\usepackage[%
	pdftitle={\titel},
	pdfauthor={\autor},
	pdfsubject={\art},
	pdfcreator={pdflatex, LaTeX with KOMA-Script},
	pdfpagemode=UseOutlines, 		% Beim Oeffnen Inhaltsverzeichnis anzeigen
	pdfdisplaydoctitle=true, 		% Dokumenttitel statt Dateiname anzeigen.
	pdflang={de}, 			% Sprache des Dokuments.
]{hyperref}

% Workaround um Fehler in Hyperref, muss hier stehen bleiben
\usepackage{bookmark} %nur ein latex-Durchlauf für die Aktualisierung von Verzeichnissen nötig

%schriftart
\usepackage{lmodern}

%lorem ipsum generator
\usepackage{lipsum}

%für kontinuierliche nummerierung von bildern
\usepackage{chngcntr}

\addbibresource{quellen.bib}

% (Farb-)einstellungen für die Links im PDF
\definecolor{LinkColor}{HTML}{00007A}
\hypersetup{%
	colorlinks=true, 		% Aktivieren von farbigen Links im Dokument
	linkcolor=LinkColor, 	% Farbe festlegen
	citecolor=LinkColor,
	filecolor=LinkColor,
	menucolor=LinkColor,
	urlcolor=LinkColor,
	linktocpage=true, 		% Nicht der Text sondern die Seitenzahlen in Verzeichnissen klickbar
	bookmarksnumbered=true 	% Überschriftsnummerierung im PDF Inhalt anzeigen.
}

% Schriftart in Captions etwas kleiner
\addtokomafont{caption}{\small}


% Hurenkinder und Schusterjungen verhindern
% http://projekte.dante.de/DanteFAQ/Silbentrennung
\clubpenalty = 10000 % schließt Schusterjungen aus (Seitenumbruch nach der ersten Zeile eines neuen Absatzes)
\widowpenalty = 10000 % schließt Hurenkinder aus (die letzte Zeile eines Absatzes steht auf einer neuen Seite)
\displaywidowpenalty=10000

% Bildpfad
\graphicspath{{Bilder/}}

%für das Quellcodeverzeichnis
\renewcommand\lstlistingname{Quellcode}
\renewcommand\lstlistlistingname{Quellcodeverzeichnis}
\counterwithout{figure}{chapter}

\begin{document}

    % Deckblatt
    \begin{spacing}{1}
        %!TEX root = arbeit.tex

\begin{titlepage}
	\enlargethispage{20mm}
	\begin{center}
		\vspace*{12mm}	{\LARGE\textbf \titel}\\
		\vspace*{12mm}	{\large\textbf \art}\\
		\vspace*{12mm}	{des Studiengangs \studienbereich}\\
		\vspace*{3mm}	{an der Dualen Hochschule Baden-Württemberg Stuttgart}\\
		\vspace*{12mm}	{von}\\
		\vspace*{3mm}		{\large\textbf \autor}\\
		\vspace*{12mm}	\today\\
	\end{center}
	\vfill
	\begin{spacing}{1.2}
	\begin{tabbing}
		mmmmmmmmmmmmmmmmmmmmmmmmmm		\= \kill
		\textbf{Bearbeitungszeitraum}	\>  \bearbeitungszeitraum\\
		\textbf{Matrikelnummer, Kurs}	\>  \matrikelnr, \kurs\\
		\textbf{Betreuer}				\>  \betreuer\\
	\end{tabbing}
	\end{spacing}
\end{titlepage}
    \end{spacing}
    \newpage

    \pagenumbering{Roman}

	% Erklärung
	%!TEX root = ../Studienarbeit.tex

\thispagestyle{empty}

\section*{Erklärung}
\vspace*{2em}

Ich versichere hiermit, dass ich meine {\art} mit dem Thema: {\itshape \titel } selbstständig verfasst und keine anderen als die angegebenen Quellen und Hilfsmittel benutzt habe. Ich versichere zudem, dass die eingereichte elektronische Fassung mit der gedruckten Fassung übereinstimmt, falls beide Fassungen gefordert sind. 

\vspace{3em}

Stuttgart, \today
\vspace{4em}

\rule{6cm}{0.4pt}\\
\autor
	\newpage

    % Abstract
	%!TEX root = ../Studienarbeit.tex

\pagestyle{empty}

\newenvironment{abstractpage}
  {\cleardoublepage\vspace*{\fill}\thispagestyle{empty}}
  {\vfill\cleardoublepage}
\newenvironment{abstractsection}[1]
  {\bigskip
   \begin{center}\bfseries#1\end{center}}
  {\par\bigskip}

\begin{abstractpage}
    \begin{abstractsection}{Kurzfassung}
      Ein wichtiger Bestandteil neben dem Multikopterfliegen stellt für Renn- und Freestyle-Quadrokopterpiloten das Training dar, welches in zwei Varianten durchgeführt werden kann. Einerseits am Flugplatz -- wobei durch Abstürze hohe Repaturkosten entstehen können. Andererseits kostengünstig im Simulator -- am Rechner. Während des Trainings am Simulator ist es wünschenswert die gewohnte Fernsteuerung zu verwenden, um den Piloten eine gewohnte Umgebung zu bieten. Um dies zu erreichen, wird in dieser Arbeit ein \acs{BLE}-Erweiterungsmodul für Multikopterfernsteuerungen entwickelt, womit der Pilot ohne weitere Treiber die Fernsteuerung mit einem Simulator auf einem Endgerät verbinden kann. Damit die Verbindung ohne Treiber funktioniert, werden die Fernsteuerungsdaten mittels \acs{HOGP} verpackt. Ebenso werden Platinen konstruiert, mit denen das Erweiterungsmodul kompakt, mobil und benutzerfreundlich verwendet werden kann. Damit die entwickelten Platinen fest und kompakt im Modulschacht (Typ: Lite) befestigt werden können, wird zusätzlich noch ein Gehäuse entwickelt, welches mittels eines \acs{FDM}-3D-Druckers gedruckt werden kann. Zuletzt wird noch die Latenz zwischen der integrierten USB-Verbindung der Fernsteuerung und der \acs{BLE}-Verbindung des Erweiterungsmoduls verglichen. Dabei kann festgestellt werden, dass unter optimalen Bedingungen die \acs{BLE}-Latenz durchschnittlich 20,84~ms länger ist als die USB-Latenz. 
    \end{abstractsection}

    \begin{abstractsection}{Abstract}
      \todo[inline]{TODO: Abstract}
    \end{abstractsection}
\end{abstractpage}
	\newpage

    \pagestyle{plain}		% nur Seitenzahlen im Fuß
	\RedeclareSectionCommand[beforeskip=20pt]{chapter} % stellt Abstand vor Kapitelüberschriften ein

	% Inhaltsverzeichnis
	\begin{spacing}{1.1}
		\begingroup
		
			% auskommentieren für Seitenzahlen unter Inhaltsverzeichnis
			\renewcommand*{\chapterpagestyle}{empty}
			\pagestyle{empty}
			
			
			\setcounter{tocdepth}{1}
			%für die Anzeige von Unterkapiteln im Inhaltsverzeichnis
			\setcounter{tocdepth}{2}
			
			\tableofcontents
			\clearpage
		\endgroup
	\end{spacing}
	\newpage

    % Abkürzungsverzeichnis
	\cleardoublepage
	%!TEX root = ../Studienarbeit.tex

\addchap{Abkürzungsverzeichnis}
%nur verwendete Akronyme werden letztlich im Abkürzungsverzeichnis des Dokuments angezeigt
%Verwendung: 
%		\ac{Abk.}   --> fügt die Abkürzung ein, beim ersten Aufruf wird zusätzlich automatisch die ausgeschriebene Version davor eingefügt bzw. in einer Fußnote (hierfür muss in header.tex \usepackage[printonlyused,footnote]{acronym} stehen) dargestellt
%		\acs{Abk.}   -->  fügt die Abkürzung ein
%		\acf{Abk.}   --> fügt die Abkürzung UND die Erklärung ein
%		\acl{Abk.}   --> fügt nur die Erklärung ein
%		\acp{Abk.}  --> gibt Plural aus (angefügtes 's'); das zusätzliche 'p' funktioniert auch bei obigen Befehlen
%	siehe auch: http://golatex.de/wiki/%5Cacronym
%	
\begin{acronym}[YTMMM]

    \acro{ADC}{Analog-Digital-Wandler}
    \acro{API}{application programming interface}
    \acro{ATT}{Attribute Protocol}
    \acro{BBR}{Bluetooth Basic Rate}
    \acro{BLE}{Bluetooth Low Energy}
    \acro{CE}{conformit\'e europ\'eenne}
    \acro{CID}{Kanalidentifizierer}
    \acro{CPU}{central processing unit}
    \acro{ESP-IDF}{Espressif IoT Development Framework}
    \acro{evdev}{Event Device}
    \acro{FCC}{Federal Communications Commission}
    \acro{GAP}{Generic Access Profile}
    \acro{GATT}{Generic Attribute Profile}
    \acro{GPIO}{general purpose input/output}
    \acro{HCI}{Host Controller Interface}
    \acro{HID}{Human Interface Device}
    \acro{HOGP}{\acs{HID} over \acs{GATT} Profile}
    \acro{IOCTL}{input output Control}
    \acro{ISM}{Industrial, Scientific and Medical}
    \acro{I2C}[I\textsuperscript{2}C]{inter integrated circuit}
    \acro{L2CAP}{Logical Link Control and Adaption Protocol}
    \acro{LL}{Link Layer}
    \acro{MFi}{Made for iPhone/iPad/IPad}
    \acro{PHY}{Physical Layer}
    \acro{PPM}{Puls-Positions-Modulation}
    \acro{RED}{Radio Equipment Directive}
    \acro{SAR}{successive approximation}
    \acro{SDP}{Service Discovery Protocol}
    \acro{SIG}{Special Interest Group}
    \acro{SMP}{Security Manager Protocol}
    \acro{UART}{Universal Asynchronous Receiver Transmitter}
    \acro{UUID}{universal unique identifier}
    \acro{WLAN}{Wireless Local Area Network}
\end{acronym}

	% Abbildungsverzeichnis
	\cleardoublepage
	\listoffigures

	%Tabellenverzeichnis
	\cleardoublepage
	\listoftables

	% Quellcodeverzeichnis
	\cleardoublepage
	\lstlistoflistings
	\cleardoublepage

    \pagenumbering{arabic}
	
	\pagestyle{headings}		% Kolumnentitel im Kopf, Seitenzahlen im Fuß

	% Inhalt
	\foreach \i in {01,02,03,04,05,06,07,08,09,...,99} {%
		\edef\FileName{Kapitel/\i kapitel}%
			\IfFileExists{\FileName}{%
				\input{\FileName}
			}
			{%
				%file does not exist
			}
	}

	\clearpage

	% Literaturverzeichnis
	\cleardoublepage
	\printbibliography
	
	% sonstiger Anhang
	\clearpage
	\appendix
	% !TeX root = ../Studienarbeit.tex

\addchap{Anhang}
{\Large
\begin{enumerate}[label=\Alph*.]
	\item Projektablauf
\end{enumerate}
}
\pagebreak
\includepdf[pages=-]{PDFs/Projektablauf.pdf}
    
\end{document}